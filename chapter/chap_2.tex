%%%%%%%%%%%%%%%%%%%%%%%%%%%%%%%%
\section{Measurable Space}
\begin{definition}[\sig-algebra]
    Suppose $X$ is a non empty set and $\mathcal{A}$ is a set of subset of $X$. Then $\mathcal{A}$ is called  \sig-algebra
    on $X$ if the following three condition are satisfied:
    \begin{enumerate}
        \item $\emptyset\in\mathcal{A}$,
        \item if $E\in\mathcal{A}$, than $E^c\in \mathcal{A}$. \textit{(close under complementation)}
        \item if $\{E_k\}$ is a sequence of element of $\mathcal{A},$ then $\bigcup_{k=1}^\infty E_k\in\mathcal{A}$. \textit{(closed under countable union)}
    \end{enumerate}
\end{definition}
\begin{theorem}
    Suppose \A\ is a \sig-algebra on a set \XX. Then 
    \begin{enumerate}
        \item $X\in\mathcal{A}$
        \item If $\{E_k\}$ is a sequence of element of \A, then $\bigcap_{k=1}^\infty\in\mathcal{A}$. \textit{(closed under countable intersection)}
    \end{enumerate}
\end{theorem}
\begin{proof}
    Because $\emptyset \in \mathcal{A}$ and \A\ is closed under complementation so, $X\in\mathcal{A}$.\\
    Now, $E_1,E_2,\ldots$ is in \A\\ 
    then, $E_k^c\in$ \A\ for all $k=1,2,\ldots$.\\
    Then, 
    \begin{align*}
        &\bigcup_{k=1}^\infty E_k^c\in \mathcal{A}.\\
        \implies &\left(\bigcap_{k=1}^\infty E_k\right)^c \in \mathcal{A}\\
        \implies &\bigcap_{k=1}^\infty E_k\in \mathcal{A}.
    \end{align*}
    This proves the result.
\end{proof}

\begin{definition}[measurable space]
    A measurable space is an ordered pair  $(X,\mathcal{A})$, where \XX\ is a set and \A\ is a \sig-algebra on \XX.
\end{definition}

\begin{definition}[measurable set]
    An element of \A\ is called an \A-measurable set or just measurable set.
\end{definition}

%%%%%%%%%%%%%%%%%%%%%%%%%%%%%%%%%%%%%%%%%
\section{Topology Space}

\begin{definition}[Topology]
    A collection $\tau$ of subsets of non-empty set \XX\ is said to be topology in \XX\ if $\tau$ satisfied the following conditions
    \begin{enumerate}
        \item $\emptyset \in \tau \text{ and }X\in \tau$.
        \item if $V_i\in\tau$  for all $i=1,2,\ldots,n$, then $\bigcap_{n=1}^n V_n\in\tau$.
        \item if $\{V_\alpha\}$ is an arbitrary collection of members of $\tau$ (finite, countable, uncountable), then $\bigcup_\alpha V_\alpha\in \tau$.
    \end{enumerate}
\end{definition}

\begin{definition}[topology space]
    If $\tau$ is a topology in \XX, Then an ordered pair  $(X,\tau)$ is called topology space.
\end{definition}

\begin{definition}[Open set]
    The member of $\tau$ is called Open set in \XX.
\end{definition}

\begin{definition}[close set]
    $E$ is said to be closed  set if  $E^c\in\tau$.
\end{definition}



%%%%%%%%%%%%%%%%%%%%%%%%%%%%%%%%%%%%%%%
\section{Borel set}
\begin{theorem}
    \label{smallest sigma algebra}
    If \F\ is any collection of subsets of \XX, there exists a smallest \sig-algebra \B\ in \XX\ such that $\mathcal{F}\in\mathcal{B}$\\
    This \B\ is sometime called the \sig-algebra generated by \F.
\end{theorem}
\begin{proof}
    Let $\Omega$ be the family \sig-algebras \BS\ in \XX\ which contain \F. Since the collection of all subsets of \XX\ is such \sig-algebra (i.e. $P(X)$ is a \sig-algebra), $\Omega$ is non empty.\\
    Let \B\ is  intersection of all $\mathcal{B}^*\in\Omega$. It is clear that $\mathcal{F}\subset\mathcal{B}$ and \B\ lies in every \sig-algebra in \XX\ which contain \F.\\
    To Complete the proof, we have to show that \B\ is a \sig-algebra.
    
    If $A_n\in\mathcal{B}$ for $n=1,2,\ldots$ and if $\mathcal{B}^*\in \Omega$, then $A_n\in\mathcal{B}^*$ so, $\bigcup_{n=1}^\infty A_n\in \mathcal{B}^*$,
    since \BS\ is a \sig-algebra\\
    Since $\bigcup_{n=1}^\infty A_n\in \mathcal{B}^*$ for every $\mathcal{B}^*\in \Omega$, we conclude that $\bigcup_{n=1}^\infty A_n\in \mathcal{B}$. The other two properties of \sig-algebra are verified in the same manner.
\end{proof}

\begin{definition}[Borel  set]
    Let \XX\  be a topology space. By theorem \ref{smallest sigma algebra} there exist a smallest \sig-algebra \B\ in \XX\ such that every open set in \XX\ belongs to \B.  The member of \B\ is called Borel  sets of \XX.
\end{definition}

\textbf{In particular:} The \sig-algebra on \RR\ containing all open subset of \RR\ is called the collection of Borel subset of \RR. An element of this \sig-algebra is called a \textit{Borel Set}.

\begin{example}
    \
    \begin{itemize}
        \item Every closed subset of \RR\ is a Borel set because every closed set are complement of an open subset of \RR.
        \item Every countable subset of \RR\ is a Borel set because countable sets are closed set.
        \item Every half-open interval $[a,b)$ (where $a,b\in \mathds{R}$) is a Borel sets because 
        \[
        [a,b)=\bigcap_{n=1}^\infty\left(a-\frac{a}{n},b\right).
        \]
        \item All countable union of closed sets and all countable intersection of open set are Borel set. This sets are called $F_\sigma$ and $G_\delta$ respectively.
    \end{itemize}
\end{example}

