\section{Cantor Set}
Every countable set has outer measure $0$. So the natural question is the converse holds. In other word, is every set with outer measure  $0$ countable?

To construct a cantor set we start with interval $[0,1]$ and removing the middle-third open interval at each step of remaining intervals from previous step.

At first step we remove  $G_1=(\frac{1}{3},\frac{2}{3})$ then we have $[0,1]\setminus G_1=[0,\frac{1}{3}]\cup[\frac{2}{3},1]$.\\
In the second step we remove $G_1\cup G_2=(\frac{1}{3},\frac{2}{3})\cup(\frac{1}{9},\frac{2}{9})\cup(\frac{7}{9},\frac{8}{9})$. Then we left with 
\[
    \left[ 0,1 \right]\setminus (G_1\cup G_2) = \left[0,\frac{1}{9}\right]\cup\left[ \frac{2}{9},\frac{1}{3} \right] \cup \left[ \frac{2}{3},\frac{7}{9} \right] \cup\left[ \frac{8}{9},1\right] 
\]
repeating this presses we construct cantor set.

\begin{definition}[Cantor Set]
    The cantor set $C$ is  $[0,1]\setminus (\bigcup_{k=1}^{\infty}G_k)$, where $G_1=(\frac{1}{3}.\frac{2}{3})$ and $G_n$ for  $n>1$ is the union of the middle-third
    open intervals in the interval of  $[0,1]\setminus (\bigcup_{k=1}^{n-1}G_k$).
\end{definition}
\begin{figure}[!h]
    \centering
    \begin{tikzpicture}[decoration=Cantor set,line width=1.5mm]
        \draw (0,0) -- (6,0);
        \draw decorate{ (0,-1) -- (6,-1) };
        \draw decorate{ decorate{ (0,-2) -- (6,-2) }};
        \draw decorate{ decorate{ decorate{ (0,-3) -- (6,-3) }}};
        \draw decorate{ decorate{ decorate{ decorate{ (0,-4) -- (6,-4) }}}};
    \end{tikzpicture}
    \caption{Cantor Set}
\end{figure}

\begin{theorem}
    \ 
    \begin{enumerate}
        \item The Cantor set is a closed subset of $\mathds{R}$.
        \item The Cantor set has Lebesgue measure $0$.
        \item The Cantor contains no interval with more then one one element.
    \end{enumerate}
\end{theorem}
\begin{proof}
    Each set $G_n$ used in the definition of cantor set is a union of open intervals. Thus  $G_n$ is open. Thus  $\bigcup_{n=1}^{\infty}G_n$ is open set,
    and hence its complement is closed. The Cantor set equals $[0,1]\cap\left( \mathds{R}\setminus \bigcup_{n=1}^{\infty}G_n \right)$, which is the intersection of 
    two closed sets. Thus the Cantor set is a closed, completing the proof of (1).

    By induction on $n$, each  $G_n$ is the union of  $2^{(n-1)}$ disjoint open interval, each of which length  $\frac{1}{3^{n}}$. Thus $\smu(G_n)=\frac{2^{n-1}}{3^{n}}$.
    The set $G_1,G_2,\ldots$ are disjoint. Hence,
    \begin{align*}
        \smu\left( \bigcup_{n=1}^{\infty}G_n \right) &= \frac{1}{3}+\frac{2}{9}+\frac{4}{27}+\ldots\\
                                                     &= \frac{1}{3}\left( 1 +\frac{2}{3}+\frac{4}{9}+\ldots\right) \\
                                                     &=\frac{1}{3}\cdot\frac{1}{1-\frac{2}{3}}\\
                                                     &=1.
    \end{align*}
    Thus the Cantor set, which is equal to $[0,1]\setminus \bigcup_{n=1}^{\infty}G_n$, has Lebesgue measure $1-1$. In other word the Cantor has Lebesgue measure  $0$.
    This proves (2).

    A set with Lebesgue measure  $0$ cannot contain an interval that has more than one element. Thus  $(2)\implies(3)$.
\end{proof}


Now observe that 0 and 1 always belong to  Cantor Set $C$ and all the end points created in each new steps are belong to  the  $C$. 
i.e. $\frac{1}{3},\frac{2}{3}\in C$ and $\frac{1}{9},\frac{2}{9},\frac{7}{9},\frac{8}{9}\in C$, and  $\frac{1}{27},\frac{2}{27},\frac{8}{27},\ldots\in C$.\\
Clearly the numbers which has integer of power 3 in denominator are in $C$. \\
\textbf{\textit{Then  Cantor set is infinite}}.

Now, 0 is always appear in left side of every line segment in all steps. 1 is always appear in right side of every line segment in all steps.
$\frac{1}{3}$ first appear in left line segment and then right in all steps. $\frac{2}{3}$ appear in right ligne segment and then left in all steps.\\
Then we write,
\begin{center}
    \large
    \begin{align*}
        0&= L\ L\ L\ L\ L\ \ldots\\
        1&= R\ R\ R\ R\ R\ \ldots \\
        \frac{1}{3}&= L\ R\ R\ R\ R\ \ldots\\
        \frac{2}{3}&= R\ L\ L\ L\ L\ \ldots\\
        \frac{1}{9}&= L\ L\ R\ R\ R\ \ldots\\
        \frac{2}{9}&= L\ R\ L\ L\ L\ \ldots\\
        \vdots
    \end{align*}
\end{center}
By this way every element of Cantor set can be express in term of infinitely many $L$ and  $R$. In other word every infinite combination of  $L$ and  $R$ 
represent a element of  $C$.

Now, we are changing  $L$ by 0 and  $R$ by 1. Then,
\begin{center}
    \large
    \begin{align*}
        0&= 0\ 0\ 0\ 0\ 0\ \ldots\\
        1&= 1\ 1\ 1\ 1\ 1\ \ldots \\
        \frac{1}{3}&= 0\ 1\ 1\ 1\ 1\ \ldots\\
        \frac{2}{3}&= 1\ 0\ 0\ 0\ 0\ \ldots\\
        \frac{1}{9}&= 0\ 0\ 1\ 1\ 1\ \ldots\\
        \frac{2}{9}&= 0\ 1\ 0\ 0\ 0\ \ldots\\
        \vdots
    \end{align*}
\end{center}
That is ever number of cantor set can be represent by 0's and 1's. 

If possible let us consider, Cantor set is countable. Then we have a countable sequence of number of Cantor set.
Consider,
\begin{center}
    \Large
    \begin{align*}
        & \underbar{0}\ 0\ 0\ 0\ 0\ 0\ 0\ \ldots \\
        & 1\ \underbar{1}\ 1\ 1\ 1\ 1\ 1\ \ldots \\
        & 0\ 1\ \underbar{0}\ 1\ 1\ 0\ 1\ \ldots \\
        & 0\ 1\ 1\ \underbar{1}\ 0\ 1\ 0\ \ldots \\
        & 1\ 1\ 0\ 1\ \underbar{0}\ 0\ 1\ \ldots \\
        & 1\ 0\ 1\ 0\ 1\ \underbar{0}\ 1\ \ldots \\
        & 0\ 0\ 1\ 1\ 0\ 1\ \underbar{0}\ \ldots \\
        &\vdots\ \ \ \ \vdots\ \ \ \ \vdots\ \ \ \ddots 
    \end{align*}
\end{center}
By hypothesis every point of cantor set are in above sequence. But let us consider the number by changing value of $(ii)$(under line position) element in every row.
Then the new number  ($1\ 0\ 1\ 0\ 1\ 1\ 1 \ldots$) is not in above sequence of numbers because it has at least one different digit from ever element in above sequence. But ($1\ 0\ 1\ 0\ 1\ 1\ 1 \ldots$) represent a element of $C$. A contradiction.\\
\textit{\textbf{Cantor Set is uncountable}}.

In fact $C$ contain as many element as  $\mathds{R}$. Although canto set is has measure $0$.
