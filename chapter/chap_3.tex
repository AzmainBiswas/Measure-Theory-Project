\section{Measurable Function}
\begin{definition}[Measurable Function]
    Let $\left( X,\mathcal{A} \right) $ and $\left( Y,\mathcal{S} \right) $ be two measurable spaces and a function $f:X \to Y$ is said to be $\mathcal{A}$-measurable function(or just measurable function) provided that  $f^{-1}(V)$ is a measurable set in $X$ for every measurable set $V$  in $Y$.\\
    i.e.
    \[
        f^{-1}(V)\in\mathcal{A} \ \ \  \ \text{for every}\ V\in\mathcal{S}
    \]
\end{definition}
\textbf{In particular:} Suppose $(X,\mathcal{A})$ is a measurable space. A function  $f:X\to\mathds{R}$ is called $\mathcal{A}$-measurable if  $f^{-1}(B)\in\mathcal{A}$ for every Borel set $B\subset \mathds{R}$.

\begin{example}
    \
    \begin{itemize}
        \item If $\mathcal{A}$ is a set of all subset of $X$, then every function from  $X\to\mathds{R}$ is a $\mathcal{A}$-measurable.
        \item If $\mathcal{A}=\{\emptyset , (-\infty,0), [0,\infty),\mathds{R}\}$ (which is a $\sigma$-algebra of  $\mathds{R}$), then constant function \\
            $f:\mathds{R}\to\mathds{R}$ is an $\mathcal{A}$-measurable function.
    \end{itemize}
\end{example}

\begin{definition}[characteristic function]
    Suppose $E$ is a subset of  $X$. The characteristic function of $E$ is a function  $\chi_{E}:X\to\mathds{R}$ define by 
    \begin{equation*}
        \chi_{E}(X)=
        \begin{cases}
            1 \ \ \text{if}\ x\in E \\
            0 \ \ \text{if}\ x\notin E
        \end{cases}
    \end{equation*}
\end{definition}
\textbf{Observation:} Suppose $(X,\mathcal{A})$ is a measurable space,  $E\subset X$, and $B\subset\mathds{R}$. Then,
\[
    \chi_E^{-1}(B)=
    \begin{cases}
        E \ \ &\text{if}\ 0\notin B\ \text{and}\ 1\in B.\\ 
        X\setminus E \ \ &\text{if}\ 0\in B\ \text{and}\ 1\notin B.\\ 
        X \ \ &\text{if}\ 0\in B\ \text{and}\ 1\in B.\\ 
        \emptyset \ \ &\text{if}\ 0\notin B\ \text{and}\ 1\notin B.
    \end{cases}
\]
thus we see that $\chi_E$ is an  $\mathcal{A}$-measurable function iff  $E\in\mathcal{A}$.

\begin{definition}[Simple Function]
    Let $\psi$ be real-valued function defined on  $X$. If the range of  $\psi$ is finite, we say that  $\psi$ is a simple function.
\end{definition}
Suppose the range of $\psi$ consists of the distinct number  $c_1,\ldots,c_n$. Let,
\[
    E_i=\{x:\psi(x)=c_i\}\ \ \ (i=1,\ldots,n)
\]
Then,
\[
    \psi(x)=\sum_{k=1}^{n}c_k\chi_{E_{i}}.
\]
i.e. every simple function is a finite linear combination of characteristic functions. Hence it is clear that $\psi$ is measurable iff the sets  $E_1,\ldots E_n$
are measurable.

It is of interest that every function can be approximated by simple function.
\newpage
\begin{theorem}[Approximation by simple function]
    $f:(X,\mathcal{A})\to\mathds{R}^{+}$ is a non-negative measurable function. Then there exists a sequence of simple function $\{\psi_n\}$ such that.
     \begin{enumerate}
         \item $\psi$ is monotone increasing.
         \item  $\lim\psi_n=f$.
     \end{enumerate}
\end{theorem}
\begin{proof}
    Let $k$ be any integer. We write,
     \[
         E_{p,k}=\left\{ x:\frac{p-1}{2^{k}}\le f(x)\le \frac{p}{2^{k}} \right\} \ \text{where}\ p=1,2,\ldots,2^{2k}.
    \]
    and 
    \[
        E_{0,k}=X\setminus \bigcup_{p=1}^{2^{2k}}E_{p,k}=\left\{ x:f(x)\ge 2^{k} \right\}.
    \]
    Then,
    \[
        E_{p,k}=\left\{ x:f(x)<\frac{p}{2^{k}} \right\}\cap\left\{ x:f(x)\ge \frac{p-1}{2^{k}} \right\}.
    \]
    is measurable, Since $f$ is measurable.
    Also, $E_{p,k},\ p=0,1,2,\ldots,2^{2k}$ are mutually disjoint.\\
    we construct,
    \[
        \psi_k(x)=
        \begin{cases}
            \frac{p-1}{2^{k}}\ \ &\text{for} \ x\in E_{p,k}\\
            2^{k} &\text{for} \ x\in E_{0,p}.
        \end{cases}
    \]
    the $\psi_k$ is a step function.\\
    By construction $0\le \psi_k\le f$.\\
    If $x\in E_{p,k},\ p\neq 0$ then we have the following cases.
    
    \textit{Case:1}\\
    $x\in E_{2p-1,k+1}$ in that cases  $\psi_k(x)=\psi_{k+1}$
    
    \textit{Case:2}\\
    $x\in E_{2p,k+1}$ is that case $\psi_k(x)+\frac{1}{2^{k+1}}=\psi_{k+1}(x)$.\\
    Thus for $x\in E_{p,k}$.  Then $\psi_n(x)\le \psi_{n+1}(x)$.\\
    Also, if $x\in E_{0,p}$. Then $\psi_k(x)=2^{k}\le 2^{k+1}=\psi_{k+1}(x)$.
    Combining we get,
    \[
        \psi_k(x)\le \psi_{k+1}(x) \ \ \forall\ x\in X
    \]
    i.e. $\psi_k$ is monotone increasing.\\
    Next we show  $\psi_k\to f$.
    
    \textit{Case:1} $f$ is finite.\\
    then $f(x)<2^{k}$ for some $k\in\mathds{Z}^{+}$, in that case,
    \[
        0\le f(x)-\psi_k(x)<2^{k}-\frac{2^{2k}-1}{2^{k}}=\frac{1}{2^{k}}.
    \]
    which show that $\psi_k(x)\to f(x)$ as  $k\to\infty$.

    \textit{Case:2} $f=+\infty$\\
    then $\psi_k(x)=2^{k}$ and $\psi_k\to f=\infty$ as $k\to\infty$.

    Combination both we get, 
    \[
        \psi_n\to f \ \text{as}\ n\to\infty.
    \]
\end{proof}

\begin{theorem}
    Suppose $(X,\mathcal{S})$ is a measurable space and $f:X\to\mathds{R}$ is a measurable function iff 
    \[
        f^{-1}((a,\infty))=\{x\in X: f(x)>a\}\in\mathcal{S}\ \ \forall \ a\in\mathds{R}
    \]
\end{theorem}
\begin{proof}
    Let $f$ is a measurable function.\\
    Since $\forall \ a \in\mathds{R}$, $(a,\infty)$ is a Borel set of  $\mathds{R}$. $f^{-1}((a,\infty))\in \mathcal{S}$
    
    Conversely, let $f^{-1}((a,\infty))\in \mathcal{S}$\\
    We have to prove that $f$ is a measurable.\\
    let,  $\tau=\{A\subset\mathds{R}:f^{-1}(A)\in\mathcal{S}\}$\\
    We have to show every Borel subset of $\mathds{R}$ is in $\tau$. To do this first we prove $\tau$ is a  \sig-algebra on $\mathds{R}$ .

    Clearly, $\emptyset\in\tau$, because  $f^{-1}(\emptyset)=\emptyset\in \mathcal{S}$.\\
    If $A\in\tau$, then  $f^{-1}(A)\in\mathcal{S}$ ; hence,
    \[
        f^{-1}(\mathds{R}\setminus A)=X\setminus f^{-1}(A)\in \mathcal{S}.
    \]
    Thus, $\mathds{R}\setminus A\in \tau$. In other word $\tau$ is closed  under complementation.

    If $A_1,A_2,\ldots\in \tau$\\
    then, $f^{-1}(A_1),f^{-1}(A_2),\ldots\in\mathcal{S}$.\\
    hence, 
    \[
        f^{-1}(\bigcup_{k=1}^{\infty}A_k)=\bigcup_{k=1}^{\infty}f^{-1}(A_k)\in \mathcal{S}.
    \]
    Thus, $\bigcup_{k=1}^{\infty}A_k\in\tau$. In other word $\tau$ is closed under countable unions. Thus  $\tau$ is \sig-algebra on  $\mathds{R}$.
    
    By hypothesis, $\tau$ contains  $\{(a,\infty):a\in\mathds{R}\}$. Since $\tau$ is closed under complementation,  $\tau$ contains  $\{(-\infty,b]:b\in\mathds{R}\}$.
    $\tau$ also contain  $\{(a,b]:a,b\in\mathds{R}\}$, as $\tau$ is closed under countable intersection. Now  $(a,b)=\bigcup_{k=1}^{\infty}(a,b-\frac{1}{k}]$.
    and $(-\infty,b)=\bigcup_{k=1}^{\infty}(-k,b-\frac{1}{k}]$. Hence, $\tau$ contain every open intervals of  $\mathds{R}$ , and every open set is union of 
    countable number of open intervals. Then $\tau$ contain every open set in $\mathds{R}$.

    Thus the \sig-algebra $\tau$ contains the smallest  \sig-algebra on  $\mathds{R}$ that contains all open subset of $\mathds{R}$. Hence $\tau$ contains every borel set of  $\mathds{R}$. Thus $f$ is a measurable function.
\end{proof}

\begin{theorem}[]
    If $(X,\mathcal{A})$ is a measurable space and a function  $f:X\to\mathds{R}$ is measurable if any of the following holds for each $a\in\mathds{R}$,
    \begin{enumerate}
        \item $\{x:f(x)\ge a\}=f^{-1}[a,\infty)\in \mathcal{A}$.
        \item $\{x:f(x)\le a\}=f^{-1}(-\infty,a]\in \mathcal{A}$.
        \item $\{x:f(x)< a\}=f^{-1}(-\infty,a)\in \mathcal{A}$.
    \end{enumerate}
\end{theorem}

\begin{proof}
    \
    \begin{enumerate} 
        \item Let $f$ is a measurable function from  $X\to\mathds{R}$.\\
            Then, $f^{-1}((a,\infty))\in\mathcal{A}$\\
            i.e. $ \{x:f(x)>a\}\in\mathcal{A}\ \ \forall a \in\mathds{R}$.
            now,
            \[
                \left\{ x:f(x)>a+\frac{1}{k} \right\}\in\mathcal{A} 
            \]
            Then,
            \[
                \bigcap_{k=1}^{\infty}\left\{ x:f(x)>a+\frac{1}{k} \right\}=\left\{ x:f(x)\ge a \right\}\in\mathcal{A}  
            \]
            As, $\mathcal{A}$ is a \sig-algebra.

            Conversely, Let $f$ satisfy (1). Then, $\{x:f(x)\ge a\}\in\mathcal{A} \ \ \forall a\in\mathds{R}$.\\
            now, 
            \[
                \left\{ x:f(x)\ge a-\frac{1}{k} \right\}\in\mathcal{A}. 
            \]
            then,
            \[
                \bigcup_{k=1}^{\infty}\left\{ a:f(x)\ge a-\frac{1}{k} \right\}=\left\{ a:f(x)>a \right\}\in\mathcal{A}.  
            \]
            as, $\mathcal{A}$ is \sig-algebra.
            Hence, $f$ is a measurable function.
        \item $\{x:f(x)\le a\}=X\setminus\{a:f(x)>a\}\in\mathcal{A}$ as, $\mathcal{A}$ is a \sig-algebra.
        \item $\{x:f(x)<a\}=X\setminus\{x:f(x)\ge a\}\in\mathcal{A}$
    \end{enumerate}
\end{proof}

%%%%%%%%%%%%%%%%%%%%%%

\subsection{Borel measurable function}
\begin{definition}[Borel measurable function]
    A function $f:X(\subset\mathds{R})\to\mathds{R}$ is called Borel measurable if $f^{-1}(B)$ is a Borel set for every Borel set $B\subset\mathds{R}$
\end{definition}

\begin{theorem}[Continuous function is Borel measurable]
    Every continuous real-valued function defined on a Borel subset of $\mathds{R}$ is a Borel measurable function.
\end{theorem}
\begin{proof}
    Let $X\subset\mathds{R}$ is a Borel set and $f:X\to\mathds{R}$ is a continuous function.\\
    To prove that $f$ is a Borel measurable function.\\
    Since,  $f$ is continuous in  $X$,  $f^{-1}(A)$ is a open set whenever $A\cap X$ is open set. 
    Hence, if $A$ is Borel set then  $f^{-1}(A)$ is also Borel set.
    Hence, $f$ is Borel measurable function.
\end{proof}
\begin{theorem}[Increasing function is Borel measurable]
    Every increasing function define on a Borel subset of $\mathds{R}$ is a Borel measurable function.
\end{theorem}
\begin{proof}
    Suppose $X\subset\mathds{R}$ and $f:X\to\mathds{R}$ is a increasing function. To prove $f$ is a Borel measurable function.\\
    we fix $a\in\mathds{R}$. Let $b=\inf f^{-1}((a,\infty))$, Then it is easy to see that,
    \[
        f^{-1}((a,\infty))=(b,\infty)\cap X\ \text{or}\ \ f^{-1}((a,\infty))=[b,\infty)\cap X
    \]
    Either way, we can conclude that $f^{-1}((a,\infty))$ is a Borel set.\\
    Hence $f$ is a Borel measurable set.
\end{proof}
\begin{theorem}[composition of measurable functions]
    Suppose $(X,\mathcal{A}),\ (Y,\mathcal{S}),\ (Z,\mathcal{B})$ are measurable spaces.
    And $f:X\to Y$ and  $g:Y\to Z$ are  $\mathcal{A}$-measurable and $\mathcal{S}$-measurable function respectively.
    Then $g\circ f:X\to Z$ is a $\mathcal{A}$-measurable function.
\end{theorem}
\begin{proof}
    Suppose $B\in \mathcal{B}$. Then ,
    \[
        (g\circ f)^{-1}(B)=f^{-1}(g^{-1}(B)).
    \]
    Since $g$ is a measurable function,  $g^{-1}(B)\in \mathcal{S}$. Then $f^{-1}(g^{-1}(B))\in\mathcal{A}$, as $f$ is a  $\mathcal{A}$-measurable function.\\
    Thus $(g\circ f)^{-1}(B)\in \mathcal{A}$, implies that $g\circ f$ is an  $\mathcal{A}$-measurable set.
\end{proof}

\begin{corollary}
    \label{-f}
    If $f$ is measurable then  $-f,kf,|f|,f^{2}$ is measurable where $k\in\mathds{R}$ is a constant.
\end{corollary}
\begin{proof}
    If we choose $g(x)=-x$ then $g(x)=kx$ then $g(x)=|x|$ and then $g(x)=x^{2}$. Clearly all the $g(x)$'s are continuous hence measurable.\\
    Then  $g\circ f$ is measurable. Hence the result.
\end{proof}

\begin{theorem}[]
    \label{f+g}
    Suppose $(X,\mathcal{A})$ is a measurable space and $f,g:X\to\mathds{R}$ are $\mathcal{A}$-measurable. Then,
    \begin{enumerate}
        \item $f+g,f-g$ and $fg$ is  $\mathcal{A}$-measurable function.
        \item If $g(x)\neq 0$ for all  $x\in X$, Then  $\frac{f}{g}$ is an measuravle functon.
    \end{enumerate}
\end{theorem}
\begin{proof}
    We first proof $f+g$ is a measurable.\\
    Let  $r$ is a rational numbers. Then for $c\in\mathds{R}$
    \begin{equation}
        \{x:f(x)+g(x)>c\}=\bigcup_{r\in\mathds{Q}}\{x:f(x)>r\}\cap\{x:g(x)>c-r\}.
    \end{equation}
    ( \textit{Because:} If $f(x)+g(x)>c$. Hence  $(c-g(x),f(x))$ is non empty and thus it contain some ration number  $r$.\\
    This implies that $r<f(x)$ and  $c-g(x)<r$.\\
    i.e.  $f(x)>r$ and  $g(x)>c-r$.\\
    i.e.  $x \in \{x:f(x)>r\}\cap\{x:g(x)>c-r\}$ for some  $r\in\mathds{Q}$.)

    Now both $f,g$ are measurable function. Then $\{x:f(x)>r\}$ and $\{x:g(x)>c-r\}$ are measurable.\\
    Then $\{x:f(x)>r\}\cap\{x:g(x)>c-r\}$ is also measurable.\\
    Thus $\{x:f(x)+g(x)>c\}=\bigcup_{r\in\mathds{Q}}\{x:f(x)>r\}\cap\{x:g(x)>c-r\}$ is measurable.\\
    Hence $f+g$ is also measurable.

    Now if $g$ is measurable then by \refeq{-f} $-g$ is also measurable. Thuse  $f+(-g)=f-g$ is also measurable function.

    Now 
    \[
        fg=\frac{(f+g)^{2}-f^2-g^2}{2}.
    \]
    Then by \refeq{-f} fg is also measurable.

    If $g(x)\neq 0\ \forall x\in \mathds{R}$. Then the function $h(x)=\frac{1}{x}$ define on  $\mathds{R}\setminus\{0\}\to\mathds{R}$ is continuous.
    Hence $\frac{1}{g}$ is a measurable function. Then  $f\times\frac{1}{g}=\frac{f}{g}$ is a measurable function.
\end{proof}

\begin{corollary}
    If $f$ and  $g$ is measurable function. Then  $\max(f,g)$ and  $\min(f,g)$ and measurable functions.\\
    In particular,  $f^{+}=\max(f,0)$ and $f^{-}=-\min(f,0)$ is also measurable function.
\end{corollary}
\begin{proof}
    we know,
    \begin{equation}
        \max(f,g)=\frac{(f+g)+|f-g|}{2} \ \ \ \min(f,g)=\frac{(f+g)-|f-g|}{2}.
    \end{equation}
    then by \refeq{f+g} this result holds.
\end{proof}

\begin{theorem}[limits of measurable function]
    Let $\{f_n\}$ be a sequence of measurable function (with the same domain of definition).\\
    Then  $\sup\{f_1,\ldots,f_n\}$, $\inf\{f_1,\ldots,f_n\}$, $\sup_nf_n$,  $\inf_nf_n$,
     $\lim\sup f_n$ and  $\lim\inf f_n$ are measurable function.
\end{theorem}
\begin{proof}
    let, $h(x)=\sup\{f_1,\ldots,f_n\}$, then,
    \[
        \{x:h(x)<a\}=\bigcap_{i=1}^{n}\{x:f_i(x)<a\}.
    \]
    Since each $f_i$'s are measurable,  $h(x)$ is also measurable.
    
    Let, $g(x)=\inf\{f_1,\ldots,f_n\}$, then
    \[
        \{x:g(x)>a\}=\bigcap_{i=1}^{n}\{x:f_i(x)>a\}.
    \]
    Since each $f_i$'s are measurable,  $g(x)$ is also measurable.

    Now, 
    $$\sup_n f_n=\lim_{n\to\infty}\{x:h(x)<a\}=\bigcap_{i=1}^{\infty}\{x:f_i(x)<a\}$$
    $$\inf_n f_n=\lim_{n\to\infty}\{x:g(x)>a\}=\bigcap_{i=1}^{\infty}\{x:f_i(x)>a\}$$
    By measurability of $f_i$'s  $\sup_nf_n$ and  $\inf_nf_n$ is measurable.

    Now, 
    \[
        \lim\sup f_n=\inf_n\big(\sup_{k\ge n}f_k\big) \ \ \text{and} \ \lim\inf f_n=\sup_n\big(\inf_{k\ge n}f_k\big)
    \]
    Then $\lim\sup f_n$ and  $\lim\inf f_n$ is also measurable.
\end{proof}
